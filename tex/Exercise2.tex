\section{Exercise 2: Capacity estimation}

\begin{itshape}
\small
We now define the capacity of a Hopfield network of size N as the number of patterns $P_{N,max}$ that can be stored, such that the mean retrieval error averaged over all stored patterns (one retrieval attempt each) is at most 2$\%$. This yields the maximal load $\alpha_{N,max} = \frac{P_{N,max} }{N}$.

Set c=0.1. Calculate $\alpha_{N,max}$ for at least 10 network realizations and state the mean together with confidence intervals. Do this for N = 100, 250 and one other larger network size. Shortly interpret the resulting values and compare with results from literature.
\end{itshape}

\paragraph*{}

Our results for $P_{N,max}$, $\alpha_{N,max}$ are shown in table \ref{tbl:exercise2_results}.

\begin{table}[H] 
\centering 
\begin{table}[H] 
\centering 
\begin{tabular}{|l|l|l|l|l|l|l|} 
\hline 
N & tests & C-level & maximal load & lower bound & upper bound & $P_{N,max}$\\ 
 \hline \hline 
100 & 10 & 95.0 $ \% $& 0.1180 & 0.1163 & 0.1197 & 11.80 \\ 
250 & 10 & 95.0 $ \% $& 0.1252 & 0.1246 & 0.1258 & 31.30 \\ 
500 & 10 & 95.0 $ \% $& 0.1282 & 0.1279 & 0.1285 & 64.10 \\ 
\hline 
\end{tabular} 
\end{table} 

\label{tbl:exercise2_results}
\end{table} 

In the literature (\citep{Hertz1991}) we find table \ref{tbl:exercise2_lit_value}.

\begin{table}[H] 
\centering 
\begin{tabular}{|l|l|}
\hline
$P_{error}$ & $p_{max}/N$ \\ \hline \hline
0.001 & 0.105 \\
0.0036 & 0.138 \\
0.01 & 0.185 \\
0.05 & 0.37 \\
0.1 & 0.61 \\
\hline
\end{tabular}
\label{tbl:exercise2_lit_value}
\end{table} 

%p_max = 4.9477160735 * P_error + 0.1187211869 
% P_error = 0.02 : p_max = 0.22

Based on a linear expression this would mean $p_{max} = 0.22$. Therefore our values are very low. Nearly the half.